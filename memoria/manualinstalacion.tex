%%%%%%%%%%%%%%%%%%%%%% WINDOWS %%%%%%%%%%%%%%%%%%%%%%
\section{Windows.}

\paragraph{}
Para jugar a \emph{Zycars} en el sistema operativo Windows no es necesario la instalación de ningún programa
auxiliar, lo único que necesitaremos descargar será la versión correspondiente
a Windows, llamada \textbf{zycars\_1.0\_win.zip } y descomprimirla. La descargaremos
del siguiente enlace:\\

\url{http://code.google.com/p/zycars/downloads/list}

\paragraph{}
Tras descomprimir el archivo, accedemos a la carpeta generada llamada ''zycars\_1.0\_win'' y hacemos doble click 
sobre el archivo \textbf{zycars.exe} para comenzar a jugar.


%%%%%%%%%%%%%%%%%%%%% UBUNTU PAQUETE DEBIAN %%%%%%%%%%%%%%%%%%%%5
\section{Linux: Ubuntu. Desde paquete Debian.}

\paragraph{}
Para poder realizar la instalación de la aplicación desde el paquete Debian, debemos descargarnos el fichero Debian 
llamado \textbf{zycars\_1.0-1\_all.deb}. Descargamos el fichero desde el siguiente enlace:\\

\url{http://code.google.com/p/zycars/downloads/list}

\paragraph{}
Una vez completada la descarga del fichero, hacemos doble click sobre este, y nos indicará si es necesario la instalación 
de algún paquete. Cuando ya estén instaladas todas las dependencias hacemos click en instalar y esperamos a la finalización
de la instalación.

\paragraph{}
Para comenzar a jugar nos vamos a Aplicaciones -> Juegos -> zycars.

\paragraph{}
En el caso que no encontremos el juego instalado en la ruta anterior, lo podremos encontrar en Aplicaciones -> Otras -> zycars.

%%%%%%%%%%%%%%%%%%%%%% UBUNTU CÓDIGO FUENTE %%%%%%%%%%%%%%%%%%%%%%
\section{Linux: Ubuntu. Desde código fuente.}

\paragraph{}
Para poder ejecutar \emph{Zycars} desde el código fuente, será necesario la instalación de varias
dependencias, para el correcto funcionamiento de la aplicación.

\paragraph{}
La primera de las dependencias a instalar será \emph{Pygame}, que es la biblioteca principal con la que
se ha desarrollado la aplicación. Para instalar, abrimos una terminal y ejecutamos el siguiente comando:

\begin{lstlisting}[style=consola, numbers=none]
sudo apt-get install python-pygame
\end{lstlisting}

\paragraph{}
Una vez instalado \emph{Pygame}, la siguiente dependencia que instalaremos será \emph{Subversion} para poder
obtener la versión más reciente del proyecto del repositorio del mismo. Para instalar subversion ejecutamos 
la siguiente orden en una terminal:

\begin{lstlisting}[style=consola, numbers=none]
sudo apt-get install subversion
\end{lstlisting}

\paragraph{}
Tras instalar \emph{Subversion}, hacemos checkout del repositorio del proyecto. Para ello ejecutamos en la terminal:

\begin{lstlisting}[style=consola, numbers=none]
svn checkout http://zycars.googlecode.com/svn/trunk/ zycars
\end{lstlisting}

\paragraph{}
Con esto hemos obtenido la versión más reciente del código de la aplicación. Ahora accedemos a la carpeta generada
anteriormente:

\begin{lstlisting}[style=consola, numbers=none]
cd zycars/
\end{lstlisting}

\paragraph{}
Damos permisos de ejecución al fichero principal.

\begin{lstlisting}[style=consola, numbers=none]
chmod +x run_test.py
\end{lstlisting}

\paragraph{}
Tras esto ya podremos jugar sin ningún problema haciendo doble click sobre \textbf{run\_test.py} o ejecutando en la terminal:
\begin{lstlisting}[style=consola, numbers=none]
./run_test.py
\end{lstlisting}

