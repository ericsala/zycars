
\paragraph{}
En esta sección se comentarán las distintas conclusiones que se han obtenido tras la finalización del proyecto \emph{Zycars}.

\section{Resumen de objetivos}

\paragraph{}
En primer lugar comentar que es el primer proyecto de estas características al que me enfrento en solitario. Es evidente que su
realización no me ha dejado indiferente. No ha sido fácil construir una idea clara sobre lo que se quería hacer. Así como
solucionar los distintos problemas que han ido apareciendo a lo largo del desarrollo de este.

\paragraph{}
También decir que el proyecto me ha ocupado bastante más tiempo del esperado en un principio. Tuve muchos problemas y alguna que 
otra duda en algunas fases del desarrollo de proyecto, que me tuvieron bloqueado durante un tiempo hasta encontrar la solución
más adecuada para estos. A pesar de todo, estoy muy satisfecho con el resultado que se ha obtenido.

\paragraph{}
Se puede decir que el proyecto goza de buena calidad. Se ha intentado hacer un
software sencillo, intuitivo, fácil de manejar y 
entretenido para el jugador. Algo esencial para un juego de estas características, en el que se busca que cualquier persona
pueda echar algún rato de su tiempo libre y qué menos que disfrute durante ese tiempo.

\section{Conclusiones personales}

\paragraph{}
Durante el desarrollo del proyecto se han aprendido muchísimas cosas: como hacer distintas ramas de desarrollo, plantear y crear
calendarios, usar las herramientas adecuadas, hacer decisiones importantes para el desarrollo de este, documentación del 
código, organización, etc. Ya que durante la carrera se han realizado distintas prácticas y trabajos de complejidad, pero nada
con el tamaño y duración que requiere un Proyecto de fin de carrera. Una vez finalizado este creo que tengo la experiencia necesaria
para afrontar otro proyecto con buenos resultados.

\paragraph{}
Entre las distintas herramientas, \LaTeX~es una de las que he aumentado mis conocimientos durante la realización de la 
memoria y gracias al compañero Pablo Recio por la plantilla facilitada para la realización de la memoria del proyecto, que sin duda
ha evitado muchos problemas.

\paragraph{}
Puedo decir que he aprendido un nuevo lenguaje de programación, como es \emph{Python}, ya que, que mejor forma de aprender un 
nuevo lenguaje, que realizar un proyecto con este.

\paragraph{}
He aprendido a usar con bastante soltura la biblioteca \emph{Pygame}, gracias tanto a la documentación de la página oficial, como
a la traducción disponible en Loserjuegos.

\paragraph{}
En definitiva, este proyecto me ha hecho madurar como persona y estudiante. He aprendido a buscar bibliografía, opiniones en otras
personas, compartir ideas, seguir un horario, cumplir una fechas de entrega y enfrentarme a un proyecto de estas características.

\section{Mejoras y ampliaciones}

\paragraph{}
Las posibles mejoras y ampliaciones que se podrían añadir al proyecto en futuras versiones, se comentan a continuación:

\begin{itemize}
    \item \textbf{Modo de dos jugadores}: añadir un nuevo modo de juego que nos permitiera jugar contra otra persona en el mismo
    ordenador. De forma que la pantalla quedaría dividida en dos.
    
    \item \textbf{Modo en red}: también sería una buena idea añadir un modo de
    juego en el que pudiésemos jugar en red contra otros
    oponentes. Este modo sería más conveniente que el modo de dos jugadores, ya que dos personas jugando en un mismo ordenador
    puede llegar a ser incomodo.
    
    %\item \textbf{Mayor diversidad de ítems}: implementar nuevo ítems de distintos tipos dentro de los tipos prefijados, como 
    %podrían ser misiles inteligentes o ítem que afectaran a todos los demás oponentes a la vez, como que estos encogieran por
    %ejemplo. Este es un campo que en el obtendríamos muchas ideas.
    
    \item \textbf{Soporte para varias resoluciones}: añadir soporte para varias resoluciones sería algo muy cómodo para aquellas 
    personas con pantalla muy pequeñas, como pueden ser los usuarios de netbooks, o también para persona con grandes resoluciones
    que desean una ventana de juego mayor.
    
    
    %\item \textbf{Más y mejores sonidos}: debido a que no se ha tenido la ayuda de ningún técnico de sonido, este es un aspecto en el que el
    %juego escasea bastante, ya que encontrar sonidos que concordaran con la temática del juego, era un poco complicado.
    
    \item \textbf{Grabación de las mejores vueltas para repetirlas}: implementar una opción que grabara la vuelta más rápida de
    cada uno de los circuitos, almacenándolas en un fichero, y así poder visualizarlas posteriormente.
\end{itemize}
