\section{Motivación}

\paragraph{}
Mi interés por el mundo de los videojuegos, desde muy pequeño, y tras haber cursado en la carrera
la asignatura optativa de "Diseño de videojuegos", donde aprendí mucho
relacionado con el desarrollo de estos,
aumentó mi interés por este mundo y además el desarrollo de ellos. Por lo que desde entonces consideraba seriamente realizar 
como proyecto fin de carrera un videojuego.

\paragraph{}
También he de añadir que tras conocer abiertamente el mundo del Software libre, gracias a la importancia que se le presta
en la Universidad de Cádiz. Se decidió que el proyecto fuera software libre bajo licencia GPL 3. Y así cualquier persona
interesada en el desarrollo de videojuegos y en el software libre en general, pudiera usar los recursos del proyecto
libremente.

\section{Objetivos}

\paragraph{}
El objetivo del proyecto es realizar un videojuego de conducción en dos dimensiones con vista cenital\footnote{Los elementos son
vistos desde arriba}. Se podría decir que el juego tendrá tintes de juegos como Micro Machines, disponible para diversas 
plataformas, y del Mario Kart de nintendo. En el siguiente capítulo se explicará más detalladamente el proyecto.

\paragraph{}
Otro de los objetivos principales del proyecto, es la realización del juego tanto para personas que
dedican varias horas a la consecución de videojuegos, tanto para personas casuales, que dedican poco tiempo
jugando.

\paragraph{}
Por lo que ser un juego de conducción el cual no esta compuesto por ninguna historia o trama argumental, facilita que se le pueda
dedicar pequeños intervalos de tiempo o, sin embargo, dedicarle varias horas al día.

\paragraph{}
Otro de los objetivos del proyecto, es poder hacerlo ampliable, de forma que cualquier persona mediante indicaciones y manuales
pueda añadir tanto nuevos personajes, como circuitos en los que competir.

\section{Estructura del documento}

\paragraph{}
Este documento esta compuesto por las siguientes partes:

\begin{itemize}
    \item \textbf{Introducción}: pequeña descripción del proyecto, así como los objetivos y estructura del documento.
    
    \item \textbf{Descripción general}: descripción más amplia sobre el proyecto, así como todas las características relevantes
    que tendrá.
    
    \item \textbf{Planificación}: exposición de la planificación del proyecto y las distintas etapas que esta compuesto el mismo.
    
    \item \textbf{Análisis}: fase de análisis del sistema, empleando la metodología seleccionada. Se definirán los
    requisitos funcionales del sistema, diagramas de caso de uso, diagramas de secuencia y contrato de las operaciones.

    \item \textbf{Diseño}: realización del diseño del sistema, diagramas de secuencia y clases aplicadas al diseño.
    
    \item \textbf{Implementación}: aspectos más relevantes durante la implementación del proyecto. Y problemas que han aparecido 
    durante el desarrollo de este.
    
    \item \textbf{Pruebas y validaciones}: pruebas realizada a la aplicación, con el fin de comprobar su correcto funcionamiento y
    cumplimiento de las expectativas.
    
    \item \textbf{Conclusiones}: conclusiones obtenidas tras el desarrollo de la aplicación.
    
    \item \textbf{Apéndices}: 
    \begin{itemize}
        \item \textbf{Herramientas utilizadas}: explicación de todas las herramientas usadas a lo largo del desarrollo del 
        proyecto.
        \item \textbf{Manual de instalación}: manual para la correcta instalación del proyecto en el sistema.
        \item \textbf{Manual de usuario}: manual de usuario para el correcto uso de la aplicación.
        \item \textbf{Manual de para añadir nuevos personajes}: manual donde se explica los distintos pasos necesarios para añadir
        nuevos personajes al juego.
        \item \textbf{Manual de para añadir nuevos circuitos}: manual donde se explica los distintos pasos necesarios para añadir
        nuevos circuitos al juego.
    \end{itemize}
    
    \item \textbf{Bibliografía}: libros y referencias consultado durante el desarrollo del proyecto.
    
    \item \textbf{Licencia GPL 3}: texto completo sobre la licencia GPL 3, por la cual se rige el proyecto.

\end{itemize}
