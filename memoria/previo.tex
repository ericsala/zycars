% -*-previo.tex-*-
% Este fichero es parte de la plantilla LaTeX para
% la realización de Proyectos Final de Carrera, protejido
% bajo los términos de la licencia GFDL.
% Para más información, la licencia completa viene incluida en el
% fichero fdl-1.3.tex

% Copyright (C) 2009 Pablo Recio Quijano 

\section*{Agradecimientos}

Me gustaria dar las gracias a todos esos amigos que he conocido a los largo de la carrera y me han ayudado
a lo largo la misma y desarrollo de este proyecto, así como a mi familia, pareja y tutores por el apoyo 
dado durante todo el desarollo del proyecto.

\cleardoublepage

\section*{Licencia} % Por ejemplo GFDL, aunque puede ser cualquiera

Este documento ha sido liberado bajo Licencia GFDL 1.3 (GNU Free
Documentation License). Se incluyen los términos de la licencia en
inglés al final del mismo.\\

Copyright (c) 2011 José Jesús Marente FLorín.\\

Permission is granted to copy, distribute and/or modify this document under the
terms of the GNU Free Documentation License, Version 1.3 or any later version
published by the Free Software Foundation; with no Invariant Sections, no
Front-Cover Texts, and no Back-Cover Texts. A copy of the license is included in
the section entitled "GNU Free Documentation License".\\

\cleardoublepage

\section*{Notación y formato}

Cuando nos refiramos a un programa o biblioteca en concreto, utilizaremos la
notación:\\

\emph{Python}.\\

Cuando nos refiramos a un fragmento de código, usaremos la notación:
\begin{verbatim} 
Código 
\end{verbatim}

Cuando nos refiramos a algún comando introducido en la terminal, usaremos la notación:

\begin{lstlisting}[style=consola, numbers=none]
sudo apt-get install
\end{lstlisting}
%Cuando nos refiramos a un comando, o función de un lenguaje, usaremos
%la notación: \\ \comando{quicksort}.\\
