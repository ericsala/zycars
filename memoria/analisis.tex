%%%%%%%%%%%%%%%%% TOMA DE REQUISITOS %%%%%%%%%%%%%%%%%%%%%%%5
\section{Toma de requisitos}

\paragraph{}
Para la creación de cualquier producto software, es necesario establecer las distintas condiciones y necesidades que
ha de satisfacer. Seguiremos un esquema que nos permita describir los requisitos de una forma metódica y racional.
%\paragraph{}

\subsection{Requisitos de interfaces externas}

\paragraph{}
En este apartado se describirá los requisitos de conexión del software y el hardware, así como la interfaz de usuario.

\paragraph{}
La conexión entres el software y el hardware se encarga la librería \emph{SDL}, mediante el wrapper \emph{Pygame} para 
el lenguaje de programación \emph{Python}. Por lo que al ser un sistema preestablecido, no será necesario realizar el diseño,
ni el análisis, sólo haremos uso de él.

\paragraph{}
Así que pasamos a definir la interfaz entre el usuario y el videojuego. Todas las ventanas de la aplicación tendrán una 
resolución de 800x600 píxeles, siendo posible establecer el modo de pantalla completa \footnote{El modo de pantalla completa
se podrá establecer a través del menú de opciones}. A continuación se distinguen las distintas
ventanas que el usuario encontrará en el sistema:

\begin{description}
    \item \textbf{Ventana de introducción} En esta primera ventana se mostrará únicamente el logotipo del juego, situando al usuario en
    contexto para introducirlo en la ejecución de la apliación.
    
    \item \textbf{Ventana de menú principal} La ventana del menú principal muestra el menú de inicio de \emph{Zycars}, asi como 
    todas las opciones generales del juego diponibles, que son las siguientes:
        \begin{itemize}
            \item Carrera Rápida
            \item Campeonato
            \item Contrarreloj
            \item Opciones
            \item Salir
        \end{itemize}
        En este menú y en los siguiente que se describan se usará el raton para navegar por ellos y solo será necesario
        un click sobre la opción deseada para acceder a ella.
    
    \item \textbf{Ventana de opciones} Desde la ventana de opciones se podrán modificar las distintas carácteristicas de la 
    configuración del juego como el audio o controles. Esta ventana se podría dividir en tres partes diferencias que se indican
    a continuación:
        \begin{itemize}
            \item Opciones de audio: podremos modificar el volumen de efectos de sonido y música del juego. También estará la opción
            de silenciar cualquier tipo de sonido.
            
            \item Opciones de pantalla: a través de esta ventana podremos activar o desactivar el modo de pantalla completa.
            
            \item Opciones de control: en esta ventana podremos modificar los controles del juego. Tanto de dirección, lanzamiento
            de items y pausa del juego.
        \end{itemize}
    
    \item \textbf{Ventana de selección de personaje} Esta ventana será compartida por los tres modos de juego disponibles. En ella 
    podremos elegir al personaje que desearemos controlar a lo largo de las carreras. De estos jugadores se nos mostrarán sus 
    distintas habilidades.
    
    \item \textbf{Ventana de selección de circuito} Ventana compartida por el modo carrera rápida y contrarreloj. En esta ventana
    deberemos elegir el circuito en el que deseamos competir. Se nos mostrará una imagen de cada circuito que seleccionemos.
    
    \item \textbf{Ventana de selección de campeonato} Ventana muy similar a la descrita anteriormente. En ella se nos mostrarán
    todos los circuitos de cada campeonato disponible. Pero al contrario que la anterior en esta ventana elegiremos el campeonato 
    del circuito seleccionado en el momento.
    
    \item \textbf{Ventana de juego} Ventana principal de todo el juego. Mostrará la carrera actual que se esté disputando, así como
    los distinto marcadores aclaratorios sobre el estado de la carrera, como pueden ser posiciones del mismo, item actual y
    tiempos de carrera. Según la tecla índicada en los controles del menú de opciones (ESC o p) se podrá acceder al menú
    de pausa del juego.
    
    \item \textbf{Ventana de pausa} Únicamente accesible desde la ventana de juego. Esta nos permitirá detener el juego en curso, 
    siendo posible reaundar el juego, reiniciar el mismo o volver al menú principal del juego.
    
    \item \textbf{Ventana de posiciones de carrera} Ventana mostrada al terminar alguna de las carreras. En ella nos muestra el 
    resultado de la última carrera disputada. Nos pertime continuar al siguiente circuito, en el caso del modo campeonato, o seguir
    hacia el menú principal, en el modo carrera rápida. También se nos permite reniciar la última carrera disputada.
    
    \item \textbf{Ventana de posiciones de campeonato} Ventana mostrada en el modo campeonato tras la ventana de posiciones 
    de carrera, en ella se nos muestra las posiciones de los competidores en el campeonato actual.
    
    \item \textbf{Ventana de tiempos de contrarreloj} Ventana mostrada al completar algún circuito en el modo contrarreloj. En ella
    se nos muestran los distintos tiempos conseguidos a lo largo del circuito y se nos indicará si hemos batido algún record.
    Esta ventana nos permite continuar hacia el menú principal, así como reiniciar el circuito disputado.
    
\end{description}

\subsection{Requisitos funcionales}

\subsection{Requisitos de rendimiento}

\subsection{Restricciones de diseño}

\subsection{Resquisitos del sistema software}

%%%%%%%%%%%%%%%% MODELO DE CASOS DE USO %%%%%%%%%%%%%%%%%%%%%%%
\section{Modelo de casos de uso}

\subsection{Diagrama de los casos de uso}

\subsection{Descripción de los casos de uso}

%%%%%%%%%%%%%% MODELO CONCEPTUAL DE DATOS %%%%%%%%%%%%%%%%%%%5
\section{Modelo conceptual de datos}

\subsection{Diagrama de clases conceptuales}

\subsection{Modelo de comportamiento del sistema}
