
\paragraph{}
El diseño de casos de pruebas para un videojuego es algo complicado, no sólo para \emph{Zyars}, si no para la mayoría de los
distintos tipos de videojuegos existentes. Esto se debe a que estamos en un
escenario simulando como interactúan muchísimos 
elementos entre sí. Pero las pruebas son algo necesario si deseamos un software con una calidad aceptable.

\paragraph{}
Prácticamente todos los módulos que componen la aplicación han sido probados individualmente, como pueden ser aquellos módulos 
encargados de la gestión de colisiones, la inteligencia artificial, gestión de las carreras.

\paragraph{}
También se realizaron pruebas de integración, ya que había módulos, una vez probado en solitario, debían realizar distintas acciones 
junto con otros módulos. Como puede ser la búsqueda A* o el modulo de configuración.

\paragraph{}
Otras pruebas que se realizaron fueron las de jugabilidad, tanto yo como
personas ajenas al desarrollo de proyecto probaron el 
mismo ofreciendo sus opiniones sobre aspectos que deberían ser modificados o
errores que aparecían a lo largo de la ejecución
del juego.

\paragraph{}
Se hicieron pruebas sobre la interfaz a medida que se implementaban nuevos menús de juego, donde se probaban
la reacción de esto ante situaciones para las que no estaban pensado su uso.

\paragraph{}
En definitiva, la organización de los casos de pruebas fue la siguiente:

\begin{enumerate}
    \item Tras finalizar la implementación de cada módulo, se realizaban pruebas unitarias sobre estos.
    \item A medida que distintos módulo que anteriormente probados individualmente, debían colaborar entre ellos, se
    llevaban a cabo pruebas de integración.
    \item Con las distintas versiones jugables se realizaban pruebas de jugabilidad.
    \item Pruebas de interfaz sobre los distintos menús que se implementaban.
\end{enumerate}

\section{Pruebas unitarias}

\paragraph{}
Esta pruebas se realizaron junto a la fase de implementación, conforme se implementaban nuevos módulos necesarios para la aplicación
se realizaban pruebas individuales sobre estos módulos. De esta forma se buscaban todos los caminos posibles que podría dar cada 
módulo, teniendo en cuenta aquellos que fueran más predispuesto a fallos.

\paragraph{}
De esta forma todas las sentencias se ejecutaban como mínimo una vez y los posibles fallos se encontraban de una forma más sencilla.
Por lo que también eran más fácil localizar donde estaba el problema y afronta la solución de este.

\section{Pruebas de integración}

\paragraph{}
Conforme aparecían nuevos módulos, cuya implementación era necesaria y a su vez estos requerían el uso de otro módulos que 
posteriormente habían sido probados individualmente, se realizaban pruebas de integración entre dichos módulos.

\paragraph{}
Conforme se avanzaba en el desarrollo de \emph{Zycars} se realizaban pruebas de integración a mayor escala. No solo entre módulos
del mis sistema, sino entre varios sistemas del juego.

\paragraph{}
En este apartado los principales problemas se encontraron a la hora de integrar todas las pantallas del juego, como pueden ser todos
los menús, los modos de juego y la pantalla de juego en sí. Pero la resolución
de los errores que aparecían no tuvieron gran 
dificultad.

\section{Pruebas de jugabilidad}

\paragraph{}
Cada vez que estaban disponibles nuevas versiones jugables del juego, se pedían la ayuda a personas, totalmente ajenas al desarrollo
de la aplicación, que probaran las distintas demos disponibles. De esta forma cada uno de los colaboradores daba su opinión sobre
distintos aspectos como la jugabilidad, dificultad, respuesta del juego o nivel de la inteligencia artificial. Tras recopilar la
información que todos ellos proporcionaron, se procedía a realizar los ajustes
necesarios a los distintos parámetros requeridos.

\paragraph{}
Entre los distintos aspectos a probar que se le recomendaban a los
colaboradores que hicieran especial hincapié, son los siguiente:

\begin{itemize}
    \item \textbf{Colisiones con escenario}: se les pedía que comprobarán que era imposible atravesar los objetos colisionables
    que se pueden encontrar a lo largo de los circuitos.
    
    \item \textbf{Colisiones con otros competidores}: comprobación de las colisiones con los otros competidores de la carrera y la
    reacción de los coches.
    
    \item \textbf{Control de carreras}: correcto funcionamiento del sistema de control de carrera, que se controlaran 
    correctamente las posiciones de los personajes, el control de las vuelta y la consecución correcta de esta.
    
    \item \textbf{Inteligencia artificial}: los coches controlados inteligencia artificial hacen correctamente el recorriendo y 
    lanzan ítems cuando lo ven oportuno.
    
    \item \textbf{Modo carrera rápida}: la carrera se completa correctamente y muestra la posición de los distintos jugadores tal 
    y como terminó la carrera.
    
    \item \textbf{Modo contrarreloj}: Los tiempos se controlan bien y al concluir la carrera se muestra los tiempos obtenido y si se
    ha superado algún tiempo.
    
    \item \textbf{Modo campeonato}: se pasan correctamente de una carrera a otra y el control de los puntos acumulados por carrera
    se realiza correctamente.
\end{itemize}

\paragraph{}
La mayor parte de las personas que probaron el juego, opinaron que este tenía la dificultad necesaria para que el juego fuera un 
reto competir contra la inteligencia artificial, pero no hasta el punto de que fuera imposible vencer a esta. Aun así aconsejaron
adaptar las velocidad de los distintos vehículos disponibles, para que la diferencia entre estos fuera menor y la posibilidad
de adelantamientos fuera mayor. Tras realizar los cambios propuesto, los colaboradores se mostraron más contentos con los resultados.

\section{Pruebas de interfaz}

\paragraph{}
Conforme se realizaban las distintas pantallas referentes a los menús que podemos encontrar en el juego, se realizaban pruebas 
exhaustivas sobre estas. Sobre todo se comprobaban que las interfaces fueran intuitivas y claras. Así como probar a cambiar 
valores erróneos de elementos que podríamos encontrar en el menú de opciones o a la hora de seleccionar las vueltas de carrera.

\paragraph{}
Se hizo especial hincapié en las siguientes pantallas:

\begin{itemize}
    \item \textbf{Menú principal}: comprobar si se accedía correctamente a las distintas opciones disponibles.
    
    \item \textbf{Menú de opciones}: se modificaban las opciones sin problemas y sin poder añadir valor errores, tras aceptar
    los cambios, estos se aplicaban correctamente.
    
    \item \textbf{Menú de selección de personaje}: se elegía el personaje deseado correctamente.
    
    \item \textbf{Menú de selección de circuito}: se elegía correctamente el circuito deseado y la modificación del número de vueltas
    se hacia correctamente.
\end{itemize}

\paragraph{}
Se comprobó que la interfaz era bastante solida y no era necesario ningún manual de usuario para poder navegar sobre ella. El 
sistema de opciones no admite valores erróneos.
