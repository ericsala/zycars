\begin{frame}
    \frametitle{Herramientas}

        \begin{block}{Lenguaje de programación: Python}
        Oportunidad perfecta para aprender un nuevo lenguaje de programación.\\
        Durante el curso 2009/2010, se conoció bastante bien dicho lenguaje. Entre sus principales características:
            \begin{itemize}
                \item Lenguaje interpretado de alto nivel
                \item Tipado dinámico y multiplataforma
                \item Multiparadigma, soporta orientación a objetos, programación imperativa y, 
                en menor medida, programación funcional
            \end{itemize}
            Destacar que se han obtenido unos resultado muy satisfactorios y ha cumplido todas las expectativas esperadas.

        \end{block}

        \begin{center}
                \includegraphics[scale=0.3]{imagenes/logo_python.png}
        \end{center}

\end{frame}

\begin{frame}
    \frametitle{Herramientas}

        \begin{block}{Biblioteca gráfica: Pygame}
        Wrapper de la biblioteca SDL, de C/C++, para Python, por lo que tiene todas las virtudes de dicha biblioteca:
            \begin{itemize}
                \item Multiplataforma compatible con Microsoft Windows, GNU/Linux, Mac OS y QNX.
                \item Muy completa, manupilacion de imágenes 2D, y gestión de sonido, música y la entrada estándar del sistema.
                \item Usada durante el desarrollo de la asignatura de Diseño de Videojuegos, se conocen todas sus 
                características muy bien.
            \end{itemize}
        \end{block}
        
        \begin{center}
                \includegraphics[scale=0.4]{imagenes/logo_pygame.png}
        \end{center}
\end{frame}

\begin{frame}
    \frametitle{Herramientas}

        \begin{block}{Analizador de código: Pylint}
        El código implementado debía seguir un estándar uniforme y que estuviera exento de errores o signos 
        de mala calidad. Para ello se usó la herramientaPylint.\\
        Analiza el código Python en busca de errores y señales de mala calidad.
        \end{block}

        \begin{block}{Sistema de control de versiones: Subversion}
        Todo el código y recursos de Zycars está alojado en el sistema que proporciona Google Code, bajo
        el sistema de control de versiones subversion.\\
        Nos permite: 
            \begin{itemize}
                \item Control de todas las versiones
                \item Visualizar todos los cambios
                \item Volver a versiones anteriores
            \end{itemize}
        \end{block}

\end{frame}


\begin{frame}
    \frametitle{Herramientas}

        \begin{block}{Documentación del código: Doxygen}
            \begin{itemize}
                \item Permite la documentación sencilla y legible de todo el código
                \item Generando la documentación en varios formatos como puede ser HTML o PDF.
            \end{itemize}
        Para python existe la herramienta Doxypy, que nos permite usar la convención de comentarios de
        Python y adaptarlos a Doxygen.
        %por lo que nos ahorra trabajo y sigue la normativa de código Python.
        \end{block}

        \begin{block}{Editor de mapas: Tiled}
        Editor de mapas de tiles de propósito general, escrito en C++, usando la librerías gráficas QT.
            \begin{itemize}
                \item Fácil de usar
                \item Flexible para trabajar con distintos motores de juegos
                \item Software libre
            \end{itemize}
        \end{block}

\end{frame}

