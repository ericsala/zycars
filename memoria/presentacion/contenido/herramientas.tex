\begin{frame}
    \frametitle{Herramientas}

        \begin{block}{Lenguaje de programación: Python}
        Oportunidad perfecta para aprender un nuevo lenguaje de programación.\\
        Entre sus principales características:
            \begin{itemize}
                \item Sintaxis limpia y que favorece un código legible.
                \item Multiplataforma
            \end{itemize}
        Destacar que se han obtenido unos resultado muy satisfactorios y ha cumplido todas las expectativas esperadas.

        \end{block}

        \begin{block}{Biblioteca gráfica: Pygame}
        Wrapper de la biblioteca SDL, de C/C++, para Python, por lo que tiene todas las virtudes de dicha biblioteca:
            \begin{itemize}
                \item Multiplataforma compatible con Microsoft Windows, GNU/Linux, Mac OS y QNX.
                \item Muy completa (imágenes 2D, sonido, música y entrada estándar)
                %\item Usada durante la asignatura de Diseño de Videojuegos, características conocidas.
            \end{itemize}
        \end{block}

\end{frame}

\begin{frame}
    \frametitle{Herramientas}

        \begin{block}{Analizador de código: Pylint}
        Analiza el código Python en busca de errores y señales de mala calidad.\\
        La nota obtenida en el código del proyecto es de 8.25 sobre 10.
        \end{block}

        %\begin{block}{Sistema de control de versiones: Subversion}
        \begin{block}{Forja del proyecto}
        Alojado en el sistema que proporciona Google Code, bajo el sistema de control de versiones subversion.
            \begin{itemize}
                \item Pública
                \item Descargas para windows, linux y código fuente
                \item Página inicial (vídeos de demos, descripción, capturas, etc)
            \end{itemize}
        \end{block}

        \begin{block}{Documentación del código: Doxygen}
            \begin{itemize}
                \item Permite la documentación sencilla y legible de todo el código
                \item Generando en varios formatos como puede ser HTML o PDF.
            \end{itemize}
        Para python existe la herramienta Doxypy.
        %que nos permite usar la convención de comentarios de Python y adaptarlos a Doxygen.
        \end{block}

\end{frame}

